\documentclass[a4paper,12pt]{article}

% Import the deliverable package from common directory
\usepackage{../common/deliverable}

% Tell LaTeX where to find graphics files
\graphicspath{{../common/logos/}{./figures/}{../}}

\usepackage{xspace}
\usepackage{lipsum}

% Set the deliverable number (without the D prefix, it's added automatically)
\setdeliverableNumber{1.1}

% Begin document
\begin{document}

% Create the title page with the title as argument
\maketitlepage{External libraries integration}

\newpage

% Main Table using the new environment and command
\begin{deliverableTable}
    \tableEntry{Deliverable title}{External libraries integration}
    \tableEntry{Deliverable number}{D1.1}
    \tableEntry{Deliverable version}{v1}
    \tableEntry{Date of delivery}{March, 31st 2025}
    \tableEntry{Actual date of delivery}{March, 31st 2025}
    \tableEntry{Nature of deliverable}{Report}
    \tableEntry{Dissemination level}{Public}
    \tableEntry{Work Package}{WP1}
    \tableEntry{Partner responsible}{INPT}
\end{deliverableTable}

% Abstract and Keywords Section
\begin{deliverableTable}
    \tableEntry{Abstract}{This report describes the API documentation
      of the initial interface to external libraries such as MUMPS,
      PSCToolkit and GMSH}
    \tableEntry{Keywords}{API, linear algebra packages, linear solvers}
\end{deliverableTable}

\newpage

\begin{documentControl}
    \addVersion{0.1}{25/03/2025}{alfredo Buttari}{Initial draft}
    % \addVersion{0.2}{[Date]}{[Author name]}{[Description of changes]}
    % \addVersion{0.3}{[Date]}{[Author name]}{[Description of changes]}
    % \addVersion{1.0}{[Date]}{[Author name]}{Final version}
\end{documentControl}

\subsection*{{Approval Details}}
Approved by: [Name] \\
Approval Date: [Date]

\subsection*{{Distribution List}}
\begin{itemize}
    \item [] - Project Coordinators (PCs)
    \item [] - Work Package Leaders (WPLs)
    \item [] - Steering Committee (SC)
    \item [] - European Commission (EC)
\end{itemize}

\vspace*{2cm}

\disclaimer

\newpage

\tableofcontents % Automatically generated and hyperlinked Table of Contents

\newpage

\section{{Introduction}}

\subsection{Overview of the dealii-X Project and Objectives}

The dealii-X project is a pioneering initiative dedicated to developing a
high-performance and scalable computational platform based on the deal.II finite
element library. The project directly addresses the
HORIZON-EUROHPC-JU-2023-COE-03-01 topic, specifically focusing on the
``Personalised Medicine / Digital twin of the human body'' as an Exascale
Lighthouse application area. The overarching goal of dealii-X is to advance
existing pre-exascale digital twin applications for human organs, such as some
deal.II based applications dedicated to the simulation of the brain, heart,
lungs, liver, and cellular interactions, to achieve exascale readiness. 

The project aims to enable real-time simulations of intricate biological
processes, thereby contributing to personalized medicine and cutting-edge
healthcare research. Ultimately, this enhanced simulation capability holds the
potential to significantly improve medical diagnostics and treatment planning.

\subsection{Objectives of Work Package 1 (WP1)}

The main objective of Work Package 1 (WP1) is to serve as the foundation
for the dealii-X Centre of Excellence by enhancing and expanding the
capabilities of the deal.II library to address the challenges of exascale
computing and facilitate the creation of advanced digital twins of human
organs.

The key steps of WP1 include:
\begin{itemize}
\item Extending and improving the exascale capabilities of deal.II;
\item Improving pre-exascale modules of the deal.II library;
\item Developing an experimental polygonal discretization module for
  deal.II;
\item Integrating PSCToolkit within deal.II;
\item Integrating MUMPS within deal.II.
\end{itemize}

Specifically, the sub-work packages aim to:
\begin{itemize}
\item \textbf{WP1.1 (Lead RUB)}: Develop matrix-free computational
  methods optimized for GPU architectures and enhance the scalability
  of solvers;
    \item \textbf{WP1.2 (Lead UNIPI)}: Improve the gmsh API, develop a
      generalized interface for coupling operators, enhance reduced
      order modelling capabilities, integrate low-rank approximation
      methods, and develop block preconditioners;
    \item \textbf{WP1.3 (Lead SISSA)}: Introduce and parallelize
      polygonal discretization methods within deal.II and develop
      related multigrid techniques;
    \item \textbf{WP1.4 (Lead UNITOV)}: integrate PSCToolkit into
      deal.II, leveraging GPU computing and developing efficient
      preconditioners for multiphysics problems;
    \item \textbf{WP1.5 (Lead INPT)}: Integrate the MUMPS solver
      directly into deal.II for use in multigrid methods and explore
      low-rank and mixed-precision techniques;
\end{itemize}

In summary, WP1 is dedicated to developing and integrating fundamental
software components within the deal.II library and external libraries,
with a strong emphasis on enabling exascale computation for the
digital twin applications in WP2.

\subsection{Purpose and Scope of this Report (Deliverable D1.1)}

The purpose of this report is to make a comprehensive analysis of the
development that are necessary to integrate external libraries,
namely, MUMPS, PSCToolkit and GMSH, within the deal.II package. 


% \subsection{{Structure of the Document}}
% \begin{itemize}
    % \item Section \ref{sec:section2}: [Section Title]
    % \item Section \ref{sec:section3}: [Section Title]
    % \item Section \ref{sec:section4}: [Section Title]
    % \item Section \ref{sec:section5}: [Section Title]
    % \item Section \ref{sec:section6}: [Section Title]
% \end{itemize}

\newpage

\section{MUMPS}
\label{sec:section2}

\subsection{Existing interface}

The deal.II package used to support a direct MUMPS integration in the
past. This support was subsequently remove in favor of an indirect use
of MUMPS through the PETSc library. As a first step towards the
objectives of WP1 of the dealii-X project, this support was reverted,
as documented in Pull Request \#18255
(\url{https://github.com/dealii/dealii/pull/18255}) which is being
merged in the main branch. This minimalistic interface include a
number of basic tests.


\subsection{Planned activities}

The current MUMPS interface described in the previous version has
numerous limitations that make the use of MUMPS within deal.II of
little practical interest.

First of all, only the real double-precision MUMPS interface is
currently supported which prevents using MUMPS on complex problems
and/or on problems where full double-precision accuracy is not
needed. Furthermore, this will not allows experimenting with
mixed-precision strategies as expected from the dealii-X
project. Support for other arithmetics can be achieved in the same
vein as it is done for the UMFPACK sparse direct solver.

The current deal.II MUMPS interface does not support
distributed-memory parallelism. Currently the system matrix and
right-hand side(s) are entirely assembled on the master process where
the subsequent phases (symbolic analysis, factorization and
backward/forward substitution) take place; not
only this might be infeasible due to memory limitations but it
severely limits the performance of the sparse direct solver. This
extension can be achieved through an adequate use of deal.II data type
like \texttt{LinearAlgebra::distributed::Vector}.

Setting-up MUMPS internal configuration parameters is not possible in
the current deal.II MUMPS interface which, therefore, has to be
extended to enable the use of the more advanced features of the MUMPS
solver; these include the Block Low-Rank approximations, the GPU
support and numerous other parameters that allow for fine-tuning both
the performance and the numerical robustness of the solver. This
extension can be achieved by adding a \texttt{AdditionalData} object
to the constructor of the MUMPS solver class that takes the desired
values for selected solver parameters.

More generally, the current MUMPS interface must be modernized to make
it compliant with deal.II coding style and modern C++ programming
practice.


\newpage

\section{PSCToolkit}
\label{sec:section3}

\lipsum[8-9]

\subsection{{[Subsection Title]}}
\begin{itemize}[left=1em, itemsep=0pt, topsep=0pt] 
    \item \lipsum[10][1-2]
    \item \lipsum[10][3-4]
    \item \lipsum[10][5-6]
\end{itemize}

\subsection{{[Subsection Title]}}
\lipsum[11]

\newpage

\section{GMSH}
\label{sec:section4}

\lipsum[12-13]

\newpage 

\section{{Conclusion}} \label{sec:conclusion}

\lipsum[20]

\label{MyLastPage}

\end{document}

%%% Local Variables:
%%% mode: LaTeX
%%% TeX-master: t
%%% End:
